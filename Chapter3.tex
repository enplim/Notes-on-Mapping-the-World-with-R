% Options for packages loaded elsewhere
\PassOptionsToPackage{unicode}{hyperref}
\PassOptionsToPackage{hyphens}{url}
\documentclass[
]{article}
\usepackage{xcolor}
\usepackage[margin=1in]{geometry}
\usepackage{amsmath,amssymb}
\setcounter{secnumdepth}{-\maxdimen} % remove section numbering
\usepackage{iftex}
\ifPDFTeX
  \usepackage[T1]{fontenc}
  \usepackage[utf8]{inputenc}
  \usepackage{textcomp} % provide euro and other symbols
\else % if luatex or xetex
  \usepackage{unicode-math} % this also loads fontspec
  \defaultfontfeatures{Scale=MatchLowercase}
  \defaultfontfeatures[\rmfamily]{Ligatures=TeX,Scale=1}
\fi
\usepackage{lmodern}
\ifPDFTeX\else
  % xetex/luatex font selection
\fi
% Use upquote if available, for straight quotes in verbatim environments
\IfFileExists{upquote.sty}{\usepackage{upquote}}{}
\IfFileExists{microtype.sty}{% use microtype if available
  \usepackage[]{microtype}
  \UseMicrotypeSet[protrusion]{basicmath} % disable protrusion for tt fonts
}{}
\makeatletter
\@ifundefined{KOMAClassName}{% if non-KOMA class
  \IfFileExists{parskip.sty}{%
    \usepackage{parskip}
  }{% else
    \setlength{\parindent}{0pt}
    \setlength{\parskip}{6pt plus 2pt minus 1pt}}
}{% if KOMA class
  \KOMAoptions{parskip=half}}
\makeatother
\usepackage{color}
\usepackage{fancyvrb}
\newcommand{\VerbBar}{|}
\newcommand{\VERB}{\Verb[commandchars=\\\{\}]}
\DefineVerbatimEnvironment{Highlighting}{Verbatim}{commandchars=\\\{\}}
% Add ',fontsize=\small' for more characters per line
\usepackage{framed}
\definecolor{shadecolor}{RGB}{248,248,248}
\newenvironment{Shaded}{\begin{snugshade}}{\end{snugshade}}
\newcommand{\AlertTok}[1]{\textcolor[rgb]{0.94,0.16,0.16}{#1}}
\newcommand{\AnnotationTok}[1]{\textcolor[rgb]{0.56,0.35,0.01}{\textbf{\textit{#1}}}}
\newcommand{\AttributeTok}[1]{\textcolor[rgb]{0.13,0.29,0.53}{#1}}
\newcommand{\BaseNTok}[1]{\textcolor[rgb]{0.00,0.00,0.81}{#1}}
\newcommand{\BuiltInTok}[1]{#1}
\newcommand{\CharTok}[1]{\textcolor[rgb]{0.31,0.60,0.02}{#1}}
\newcommand{\CommentTok}[1]{\textcolor[rgb]{0.56,0.35,0.01}{\textit{#1}}}
\newcommand{\CommentVarTok}[1]{\textcolor[rgb]{0.56,0.35,0.01}{\textbf{\textit{#1}}}}
\newcommand{\ConstantTok}[1]{\textcolor[rgb]{0.56,0.35,0.01}{#1}}
\newcommand{\ControlFlowTok}[1]{\textcolor[rgb]{0.13,0.29,0.53}{\textbf{#1}}}
\newcommand{\DataTypeTok}[1]{\textcolor[rgb]{0.13,0.29,0.53}{#1}}
\newcommand{\DecValTok}[1]{\textcolor[rgb]{0.00,0.00,0.81}{#1}}
\newcommand{\DocumentationTok}[1]{\textcolor[rgb]{0.56,0.35,0.01}{\textbf{\textit{#1}}}}
\newcommand{\ErrorTok}[1]{\textcolor[rgb]{0.64,0.00,0.00}{\textbf{#1}}}
\newcommand{\ExtensionTok}[1]{#1}
\newcommand{\FloatTok}[1]{\textcolor[rgb]{0.00,0.00,0.81}{#1}}
\newcommand{\FunctionTok}[1]{\textcolor[rgb]{0.13,0.29,0.53}{\textbf{#1}}}
\newcommand{\ImportTok}[1]{#1}
\newcommand{\InformationTok}[1]{\textcolor[rgb]{0.56,0.35,0.01}{\textbf{\textit{#1}}}}
\newcommand{\KeywordTok}[1]{\textcolor[rgb]{0.13,0.29,0.53}{\textbf{#1}}}
\newcommand{\NormalTok}[1]{#1}
\newcommand{\OperatorTok}[1]{\textcolor[rgb]{0.81,0.36,0.00}{\textbf{#1}}}
\newcommand{\OtherTok}[1]{\textcolor[rgb]{0.56,0.35,0.01}{#1}}
\newcommand{\PreprocessorTok}[1]{\textcolor[rgb]{0.56,0.35,0.01}{\textit{#1}}}
\newcommand{\RegionMarkerTok}[1]{#1}
\newcommand{\SpecialCharTok}[1]{\textcolor[rgb]{0.81,0.36,0.00}{\textbf{#1}}}
\newcommand{\SpecialStringTok}[1]{\textcolor[rgb]{0.31,0.60,0.02}{#1}}
\newcommand{\StringTok}[1]{\textcolor[rgb]{0.31,0.60,0.02}{#1}}
\newcommand{\VariableTok}[1]{\textcolor[rgb]{0.00,0.00,0.00}{#1}}
\newcommand{\VerbatimStringTok}[1]{\textcolor[rgb]{0.31,0.60,0.02}{#1}}
\newcommand{\WarningTok}[1]{\textcolor[rgb]{0.56,0.35,0.01}{\textbf{\textit{#1}}}}
\usepackage{graphicx}
\makeatletter
\newsavebox\pandoc@box
\newcommand*\pandocbounded[1]{% scales image to fit in text height/width
  \sbox\pandoc@box{#1}%
  \Gscale@div\@tempa{\textheight}{\dimexpr\ht\pandoc@box+\dp\pandoc@box\relax}%
  \Gscale@div\@tempb{\linewidth}{\wd\pandoc@box}%
  \ifdim\@tempb\p@<\@tempa\p@\let\@tempa\@tempb\fi% select the smaller of both
  \ifdim\@tempa\p@<\p@\scalebox{\@tempa}{\usebox\pandoc@box}%
  \else\usebox{\pandoc@box}%
  \fi%
}
% Set default figure placement to htbp
\def\fps@figure{htbp}
\makeatother
\setlength{\emergencystretch}{3em} % prevent overfull lines
\providecommand{\tightlist}{%
  \setlength{\itemsep}{0pt}\setlength{\parskip}{0pt}}
\usepackage{bookmark}
\IfFileExists{xurl.sty}{\usepackage{xurl}}{} % add URL line breaks if available
\urlstyle{same}
\hypersetup{
  pdftitle={Chapter 3},
  pdfauthor={The Caveman Coder},
  hidelinks,
  pdfcreator={LaTeX via pandoc}}

\title{Chapter 3}
\author{The Caveman Coder}
\date{2025-10-27}

\begin{document}
\maketitle

\subsection{Concepts: Finding Your Ingredients and Checking
Quality}\label{concepts-finding-your-ingredients-and-checking-quality}

Where to find reliable spatial data:

\begin{itemize}
\tightlist
\item
  Natural Earth (naturlearthdata.com)

  \begin{itemize}
  \tightlist
  \item
    Global datasets at small, medium, and large scales\\
  \item
    Contains physical features like coastlines, rivers, lakes\\
  \item
    Contains cultural features like country, borders, cities, etc.
  \item
    Belongs to public domain, easy to use legally! - Format is usually
    Shapefiles\\
  \end{itemize}
\item
  OpenStreeMap (OSM) (openstreetmap.org)

  \begin{itemize}
  \tightlist
  \item
    Like wikipedia but for maps\\
  \item
    Contains user-contributed details like roads, buildings, and other
    points of interest\\
  \item
    Free to use (with proper attribution)\\
  \item
    Data quality varies by region (more mappers, more detail)\\
  \item
    Data structure can be quite complex\\
  \item
    Format may be a special OSM format or extracts (Shapefile or
    GeoPackage)\\
  \end{itemize}
\item
  GADM (gadm.org)

  \begin{itemize}
  \tightlist
  \item
    Database of Global Administrative Areas\\
  \item
    Provides administrative boundaries for nearly all countries\\
  \item
    Consistent source for administrative boundaries worldwide - Easy to
    access via the \texttt{geodata} package\\
  \item
    Boundaries might not perfectly match official national boundaries\\
  \item
    Format can be Shapefile, GeoPackage, or via \texttt{geodata}\\
  \end{itemize}
\item
  Government Geospatial Portals\\

  \begin{itemize}
  \tightlist
  \item
    data.gov, data.gov.uk, national mapping agency websites\\
  \item
    Can be very detailed\\
  \item
    Licensing terms may vary\\
  \item
    Format is highly variable, depending on the agency
  \end{itemize}
\end{itemize}

Search Strategy:

\begin{enumerate}
\def\labelenumi{\arabic{enumi}.}
\tightlist
\item
  Start with \texttt{naturalearth} or \texttt{geodata} packages if you
  need general world/country/state boundaries.\\
\item
  Consider \texttt{osmdata} package to access OpenStreetMap if you need
  detailed streets, buildings, cafes, parks, and other points of
  interest.\\
\item
  Search for ``{[}City/State/Country{]} GIS data portal'' or ``XX
  spatial data {[}Location{]}''.
\end{enumerate}

How to check the quality of your geospatial data: 1. Check the metadata.
Good spatial data should come with information on the source, date of
creation, description or abstract, CRS, scale or resolution, accuracy,
limitations, known issues, license or use constraints.

If metadata is missing, be extra critical.

If available, prefer GeoPackage (*.gpkg) for vector data. If you get a
Shapefile, make sure that you have all its component files together
especially the \texttt{.prj} file.

Ethical Considerations: When choosing data, watch out for source bias,
proper representation, and privacy.

\textbf{Being a good cartographer means being tecknicallyh skilled AND
ethically aware.}

\subsection{Hands-On: Loading and Cleaning Your Own
Data}\label{hands-on-loading-and-cleaning-your-own-data}

\textbf{Loading Vector Data}

Step 1: Load necessary packages

\begin{Shaded}
\begin{Highlighting}[]
\ControlFlowTok{if}\NormalTok{ (}\SpecialCharTok{!}\FunctionTok{requireNamespace}\NormalTok{(}\StringTok{"pacman"}\NormalTok{, }\AttributeTok{quietly =} \ConstantTok{TRUE}\NormalTok{)) \{}
  \FunctionTok{install.packates}\NormalTok{(}\StringTok{"pacman"}\NormalTok{)}
\NormalTok{\}}
\FunctionTok{library}\NormalTok{(pacman)}
\NormalTok{pacman}\SpecialCharTok{::}\FunctionTok{p\_load}\NormalTok{(sf, dplyr)}
\end{Highlighting}
\end{Shaded}

Step 2. Load the GeoPackage file using \texttt{st\_read()}

\begin{Shaded}
\begin{Highlighting}[]
\NormalTok{world\_boundaries\_loaded }\OtherTok{\textless{}{-}}\NormalTok{ sf}\SpecialCharTok{::}\FunctionTok{st\_read}\NormalTok{(}\StringTok{"data/world\_boundaries.gpkg"}\NormalTok{)}
\end{Highlighting}
\end{Shaded}

\begin{verbatim}
## Reading layer `world_boundaries' from data source 
##   `/home/norman/Documents/ThirdBrain/x1_Projects/RProjects/Notes-on-Mapping-the-World-with-R/data/world_boundaries.gpkg' 
##   using driver `GPKG'
## Simple feature collection with 257 features and 4 fields
## Geometry type: MULTIPOLYGON
## Dimension:     XY
## Bounding box:  xmin: -180 ymin: -89.9 xmax: 180 ymax: 83.65872
## Geodetic CRS:  WGS 84
\end{verbatim}

Step 3: Inspect the loaded data

\begin{Shaded}
\begin{Highlighting}[]
\FunctionTok{class}\NormalTok{(world\_boundaries\_loaded)}
\end{Highlighting}
\end{Shaded}

\begin{verbatim}
## [1] "sf"         "data.frame"
\end{verbatim}

\begin{Shaded}
\begin{Highlighting}[]
\FunctionTok{head}\NormalTok{(world\_boundaries\_loaded)}
\end{Highlighting}
\end{Shaded}

\begin{verbatim}
## Simple feature collection with 6 features and 4 fields
## Geometry type: MULTIPOLYGON
## Dimension:     XY
## Bounding box:  xmin: -63.1112 ymin: 16.99742 xmax: 74.88986 ymax: 42.65309
## Geodetic CRS:  WGS 84
##   CNTR_ID                CNTR_NAME ISO3_CODE            NAME_ENGL
## 1      AD                  Andorra       AND              Andorra
## 2      AE الإمارات العربية المتحدة       ARE United Arab Emirates
## 3      AF      افغانستان-افغانستان       AFG          Afghanistan
## 4      AG      Antigua and Barbuda       ATG  Antigua and Barbuda
## 5      AI                 Anguilla       AIA             Anguilla
## 6      AL                Shqipëria       ALB              Albania
##                             geom
## 1 MULTIPOLYGON (((1.53521 42....
## 2 MULTIPOLYGON (((56.1464 25....
## 3 MULTIPOLYGON (((71.28622 38...
## 4 MULTIPOLYGON (((-61.70749 1...
## 5 MULTIPOLYGON (((-63.08349 1...
## 6 MULTIPOLYGON (((19.75862 42...
\end{verbatim}

\begin{Shaded}
\begin{Highlighting}[]
\NormalTok{sf}\SpecialCharTok{::}\FunctionTok{st\_crs}\NormalTok{(world\_boundaries\_loaded)}
\end{Highlighting}
\end{Shaded}

\begin{verbatim}
## Coordinate Reference System:
##   User input: WGS 84 
##   wkt:
## GEOGCRS["WGS 84",
##     ENSEMBLE["World Geodetic System 1984 ensemble",
##         MEMBER["World Geodetic System 1984 (Transit)"],
##         MEMBER["World Geodetic System 1984 (G730)"],
##         MEMBER["World Geodetic System 1984 (G873)"],
##         MEMBER["World Geodetic System 1984 (G1150)"],
##         MEMBER["World Geodetic System 1984 (G1674)"],
##         MEMBER["World Geodetic System 1984 (G1762)"],
##         MEMBER["World Geodetic System 1984 (G2139)"],
##         MEMBER["World Geodetic System 1984 (G2296)"],
##         ELLIPSOID["WGS 84",6378137,298.257223563,
##             LENGTHUNIT["metre",1]],
##         ENSEMBLEACCURACY[2.0]],
##     PRIMEM["Greenwich",0,
##         ANGLEUNIT["degree",0.0174532925199433]],
##     CS[ellipsoidal,2],
##         AXIS["geodetic latitude (Lat)",north,
##             ORDER[1],
##             ANGLEUNIT["degree",0.0174532925199433]],
##         AXIS["geodetic longitude (Lon)",east,
##             ORDER[2],
##             ANGLEUNIT["degree",0.0174532925199433]],
##     USAGE[
##         SCOPE["Horizontal component of 3D system."],
##         AREA["World."],
##         BBOX[-90,-180,90,180]],
##     ID["EPSG",4326]]
\end{verbatim}

\begin{Shaded}
\begin{Highlighting}[]
\CommentTok{\# Make the object variable available outside the chunk if needed}
\FunctionTok{assign}\NormalTok{(}
  \StringTok{"world\_boundaries\_loaded"}\NormalTok{,}
\NormalTok{  world\_boundaries\_loaded,}
  \AttributeTok{envir =}\NormalTok{ .GlobalEnv}
\NormalTok{)}
\end{Highlighting}
\end{Shaded}

\textbf{Loading Tabular Data}

Step 1: Loading necessary packages

\begin{Shaded}
\begin{Highlighting}[]
\NormalTok{pacman}\SpecialCharTok{::}\FunctionTok{p\_load}\NormalTok{(readr, dplyr)}
\end{Highlighting}
\end{Shaded}

Step 2: Load the CSV file using \texttt{read\_csv()}

\begin{Shaded}
\begin{Highlighting}[]
\NormalTok{country\_indicators\_loaded }\OtherTok{\textless{}{-}}\NormalTok{ readr}\SpecialCharTok{::}\FunctionTok{read\_csv}\NormalTok{(}\StringTok{"data/country\_indicators.csv"}\NormalTok{)}
\end{Highlighting}
\end{Shaded}

\begin{verbatim}
## Rows: 266 Columns: 7
## -- Column specification --------------------------------------------------------
## Delimiter: ","
## chr (3): country, iso2c, iso3c
## dbl (4): year, gdp_per_capita, life_expectancy, population
## 
## i Use `spec()` to retrieve the full column specification for this data.
## i Specify the column types or set `show_col_types = FALSE` to quiet this message.
\end{verbatim}

Step 3: Inspect the loaded data

\begin{Shaded}
\begin{Highlighting}[]
\FunctionTok{class}\NormalTok{(country\_indicators\_loaded)}
\end{Highlighting}
\end{Shaded}

\begin{verbatim}
## [1] "spec_tbl_df" "tbl_df"      "tbl"         "data.frame"
\end{verbatim}

\begin{Shaded}
\begin{Highlighting}[]
\FunctionTok{head}\NormalTok{(country\_indicators\_loaded)}
\end{Highlighting}
\end{Shaded}

\begin{verbatim}
## # A tibble: 6 x 7
##   country            iso2c iso3c  year gdp_per_capita life_expectancy population
##   <chr>              <chr> <chr> <dbl>          <dbl>           <dbl>      <dbl>
## 1 Afghanistan        AF    AFG    2020           511.            61.5   39068979
## 2 Africa Eastern an~ ZH    AFE    2020          1344.            63.8  694446100
## 3 Africa Western an~ ZI    AFW    2020          1664.            57.4  474569351
## 4 Albania            AL    ALB    2020          5371.            77.8    2837849
## 5 Algeria            DZ    DZA    2020          3744.            73.3   44042091
## 6 American Samoa     AS    ASM    2020         14489.            72.7      49761
\end{verbatim}

\begin{Shaded}
\begin{Highlighting}[]
\FunctionTok{glimpse}\NormalTok{(country\_indicators\_loaded)}
\end{Highlighting}
\end{Shaded}

\begin{verbatim}
## Rows: 266
## Columns: 7
## $ country         <chr> "Afghanistan", "Africa Eastern and Southern", "Africa ~
## $ iso2c           <chr> "AF", "ZH", "ZI", "AL", "DZ", "AS", "AD", "AO", "AG", ~
## $ iso3c           <chr> "AFG", "AFE", "AFW", "ALB", "DZA", "ASM", "AND", "AGO"~
## $ year            <dbl> 2020, 2020, 2020, 2020, 2020, 2020, 2020, 2020, 2020, ~
## $ gdp_per_capita  <dbl> 510.7871, 1344.0810, 1664.2492, 5370.7786, 3743.5420, ~
## $ life_expectancy <dbl> 61.45400, 63.76648, 57.36443, 77.82400, 73.25700, 72.6~
## $ population      <dbl> 39068979, 694446100, 474569351, 2837849, 44042091, 497~
\end{verbatim}

\begin{Shaded}
\begin{Highlighting}[]
\CommentTok{\# Make the object available outside the chunk if needed}
\FunctionTok{assign}\NormalTok{(}
  \StringTok{"country\_indicators\_loaded"}\NormalTok{,}
\NormalTok{  country\_indicators\_loaded,}
  \AttributeTok{envir =}\NormalTok{ .GlobalEnv}
\NormalTok{)}
\end{Highlighting}
\end{Shaded}

\textbf{Basic Data Cleaning with \texttt{dplyr}}

Step 1: Ensure packages and data are ready

\begin{Shaded}
\begin{Highlighting}[]
\NormalTok{pacman}\SpecialCharTok{::}\FunctionTok{p\_load}\NormalTok{(sf, dplyr, rnaturalearth)}
\end{Highlighting}
\end{Shaded}

Step 2: Select only needed columns. For this exercise, we only need the
name, iso code, and geometry for our map.

\begin{Shaded}
\begin{Highlighting}[]
\CommentTok{\# world\_boundaries\_loaded \textless{}{-} sf::st\_read("world\_boundaries.gpkg")}
\NormalTok{sf}\SpecialCharTok{::}\FunctionTok{st\_geometry}\NormalTok{(world\_boundaries\_loaded) }\OtherTok{\textless{}{-}} \StringTok{"geometry"}
\NormalTok{world\_selected }\OtherTok{\textless{}{-}}\NormalTok{ world\_boundaries\_loaded }\SpecialCharTok{|\textgreater{}} 
\NormalTok{  dplyr}\SpecialCharTok{::}\FunctionTok{select}\NormalTok{(}
\NormalTok{    dplyr}\SpecialCharTok{::}\FunctionTok{any\_of}\NormalTok{(}
      \FunctionTok{c}\NormalTok{(}\StringTok{"NAME\_ENGL"}\NormalTok{, }\StringTok{"ISO3\_CODE"}\NormalTok{)}
\NormalTok{    ), geometry}
\NormalTok{  )}
\FunctionTok{head}\NormalTok{(world\_selected)}
\end{Highlighting}
\end{Shaded}

\begin{verbatim}
## Simple feature collection with 6 features and 2 fields
## Geometry type: MULTIPOLYGON
## Dimension:     XY
## Bounding box:  xmin: -63.1112 ymin: 16.99742 xmax: 74.88986 ymax: 42.65309
## Geodetic CRS:  WGS 84
##              NAME_ENGL ISO3_CODE                       geometry
## 1              Andorra       AND MULTIPOLYGON (((1.53521 42....
## 2 United Arab Emirates       ARE MULTIPOLYGON (((56.1464 25....
## 3          Afghanistan       AFG MULTIPOLYGON (((71.28622 38...
## 4  Antigua and Barbuda       ATG MULTIPOLYGON (((-61.70749 1...
## 5             Anguilla       AIA MULTIPOLYGON (((-63.08349 1...
## 6              Albania       ALB MULTIPOLYGON (((19.75862 42...
\end{verbatim}

Step 3: Rename columns for easier use

\begin{Shaded}
\begin{Highlighting}[]
\NormalTok{world\_renamed }\OtherTok{\textless{}{-}}\NormalTok{ world\_selected }\SpecialCharTok{|\textgreater{}} 
\NormalTok{  dplyr}\SpecialCharTok{::}\FunctionTok{rename}\NormalTok{(}
    \AttributeTok{country\_name =}\NormalTok{ NAME\_ENGL,}
    \AttributeTok{iso3 =}\NormalTok{ ISO3\_CODE}
\NormalTok{  )}

\FunctionTok{head}\NormalTok{(world\_renamed)}
\end{Highlighting}
\end{Shaded}

\begin{verbatim}
## Simple feature collection with 6 features and 2 fields
## Geometry type: MULTIPOLYGON
## Dimension:     XY
## Bounding box:  xmin: -63.1112 ymin: 16.99742 xmax: 74.88986 ymax: 42.65309
## Geodetic CRS:  WGS 84
##           country_name iso3                       geometry
## 1              Andorra  AND MULTIPOLYGON (((1.53521 42....
## 2 United Arab Emirates  ARE MULTIPOLYGON (((56.1464 25....
## 3          Afghanistan  AFG MULTIPOLYGON (((71.28622 38...
## 4  Antigua and Barbuda  ATG MULTIPOLYGON (((-61.70749 1...
## 5             Anguilla  AIA MULTIPOLYGON (((-63.08349 1...
## 6              Albania  ALB MULTIPOLYGON (((19.75862 42...
\end{verbatim}

Step 4: Filter rows, e.g., keep only African countries. We will get the
continent info from rnaturalearth data

\begin{Shaded}
\begin{Highlighting}[]
\NormalTok{world\_ne }\OtherTok{\textless{}{-}}\NormalTok{ rnaturalearth}\SpecialCharTok{::}\FunctionTok{ne\_countries}\NormalTok{(}
  \AttributeTok{scale =} \StringTok{"medium"}\NormalTok{,}
  \AttributeTok{returnclass =} \StringTok{"sf"}
\NormalTok{) }\SpecialCharTok{|\textgreater{}} 
\NormalTok{  dplyr}\SpecialCharTok{::}\FunctionTok{select}\NormalTok{(}
\NormalTok{    adm0\_a3,}
\NormalTok{    continent}
\NormalTok{  ) }\SpecialCharTok{|\textgreater{}} 
\NormalTok{  sf}\SpecialCharTok{::}\FunctionTok{st\_drop\_geometry}\NormalTok{()}
\end{Highlighting}
\end{Shaded}

Now join the continent info to the renamed boundaries

\begin{Shaded}
\begin{Highlighting}[]
\NormalTok{world\_renamed\_with\_cont }\OtherTok{\textless{}{-}}\NormalTok{ world\_renamed }\SpecialCharTok{|\textgreater{}} 
\NormalTok{  dplyr}\SpecialCharTok{::}\FunctionTok{left\_join}\NormalTok{(world\_ne, }\AttributeTok{by =} \FunctionTok{c}\NormalTok{(}\StringTok{"iso3"} \OtherTok{=} \StringTok{"adm0\_a3"}\NormalTok{))}
\end{Highlighting}
\end{Shaded}

Filtering for Africa:

\begin{Shaded}
\begin{Highlighting}[]
\NormalTok{africa\_only }\OtherTok{\textless{}{-}}\NormalTok{ world\_renamed\_with\_cont }\SpecialCharTok{|\textgreater{}} 
\NormalTok{  dplyr}\SpecialCharTok{::}\FunctionTok{filter}\NormalTok{(continent }\SpecialCharTok{==} \StringTok{"Africa"}\NormalTok{)}

\NormalTok{africa\_only}
\end{Highlighting}
\end{Shaded}

\begin{verbatim}
## Simple feature collection with 51 features and 3 fields
## Geometry type: MULTIPOLYGON
## Dimension:     XY
## Bounding box:  xmin: -25.35984 ymin: -46.98082 xmax: 51.41287 ymax: 37.34131
## Geodetic CRS:  WGS 84
## First 10 features:
##                        country_name iso3 continent
## 1                            Angola  AGO    Africa
## 2                          Botswana  BWA    Africa
## 3                             Congo  COG    Africa
## 4                     Côte D’Ivoire  CIV    Africa
## 5                      Burkina Faso  BFA    Africa
## 6                           Burundi  BDI    Africa
## 7                             Benin  BEN    Africa
## 8  Democratic Republic of The Congo  COD    Africa
## 9          Central African Republic  CAF    Africa
## 10                          Algeria  DZA    Africa
##                          geometry
## 1  MULTIPOLYGON (((13.47062 -5...
## 2  MULTIPOLYGON (((25.16481 -1...
## 3  MULTIPOLYGON (((18.49177 3....
## 4  MULTIPOLYGON (((-5.663248 1...
## 5  MULTIPOLYGON (((-0.7119468 ...
## 6  MULTIPOLYGON (((30.41149 -2...
## 7  MULTIPOLYGON (((3.484148 11...
## 8  MULTIPOLYGON (((27.41356 5....
## 9  MULTIPOLYGON (((22.50105 11...
## 10 MULTIPOLYGON (((8.442687 36...
\end{verbatim}

Step 5: Check and filter out invalid geometries

\begin{Shaded}
\begin{Highlighting}[]
\CommentTok{\# Count initial rows}
\NormalTok{initial\_rows }\OtherTok{\textless{}{-}} \FunctionTok{nrow}\NormalTok{(africa\_only)}

\CommentTok{\# Remove features where geometry is empty}
\NormalTok{africa\_valid\_geom }\OtherTok{\textless{}{-}}\NormalTok{ africa\_only }\SpecialCharTok{|\textgreater{}} 
\NormalTok{  dplyr}\SpecialCharTok{::}\FunctionTok{filter}\NormalTok{(}\SpecialCharTok{!}\NormalTok{sf}\SpecialCharTok{::}\FunctionTok{st\_is\_empty}\NormalTok{(geometry))}

\CommentTok{\# Count final rows}
\NormalTok{final\_rows }\OtherTok{\textless{}{-}} \FunctionTok{nrow}\NormalTok{(africa\_valid\_geom)}

\FunctionTok{print}\NormalTok{(}\FunctionTok{paste}\NormalTok{(}\StringTok{"Initial Africa rows:"}\NormalTok{, initial\_rows))}
\end{Highlighting}
\end{Shaded}

\begin{verbatim}
## [1] "Initial Africa rows: 51"
\end{verbatim}

\begin{Shaded}
\begin{Highlighting}[]
\FunctionTok{print}\NormalTok{(}\FunctionTok{paste}\NormalTok{(}\StringTok{"Rows after removing/fixing empty/invalid rows:"}\NormalTok{, final\_rows))}
\end{Highlighting}
\end{Shaded}

\begin{verbatim}
## [1] "Rows after removing/fixing empty/invalid rows: 51"
\end{verbatim}

\textbf{Joining Tabular Data to Spatial Data}

Step 1: Load packages and verify data exists

\begin{Shaded}
\begin{Highlighting}[]
\NormalTok{pacman}\SpecialCharTok{::}\FunctionTok{p\_load}\NormalTok{(sf, dplyr, readr)}
\end{Highlighting}
\end{Shaded}

Step 2: Load the dataset aand inspect key columns

\begin{Shaded}
\begin{Highlighting}[]
\NormalTok{country\_indicators\_loaded }\OtherTok{\textless{}{-}}\NormalTok{ readr}\SpecialCharTok{::}\FunctionTok{read\_csv}\NormalTok{(}\StringTok{"data/country\_indicators.csv"}\NormalTok{)}
\end{Highlighting}
\end{Shaded}

\begin{verbatim}
## Rows: 266 Columns: 7
## -- Column specification --------------------------------------------------------
## Delimiter: ","
## chr (3): country, iso2c, iso3c
## dbl (4): year, gdp_per_capita, life_expectancy, population
## 
## i Use `spec()` to retrieve the full column specification for this data.
## i Specify the column types or set `show_col_types = FALSE` to quiet this message.
\end{verbatim}

\begin{Shaded}
\begin{Highlighting}[]
\FunctionTok{head}\NormalTok{(world\_renamed}\SpecialCharTok{$}\NormalTok{iso3)}
\end{Highlighting}
\end{Shaded}

\begin{verbatim}
## [1] "AND" "ARE" "AFG" "ATG" "AIA" "ALB"
\end{verbatim}

\begin{Shaded}
\begin{Highlighting}[]
\FunctionTok{head}\NormalTok{(country\_indicators\_loaded}\SpecialCharTok{$}\NormalTok{iso3c)}
\end{Highlighting}
\end{Shaded}

\begin{verbatim}
## [1] "AFG" "AFE" "AFW" "ALB" "DZA" "ASM"
\end{verbatim}

Step 3: Perform the left join

\begin{Shaded}
\begin{Highlighting}[]
\NormalTok{world\_data\_joined }\OtherTok{\textless{}{-}}\NormalTok{ world\_renamed }\SpecialCharTok{|\textgreater{}} 
\NormalTok{  dplyr}\SpecialCharTok{::}\FunctionTok{left\_join}\NormalTok{(country\_indicators\_loaded, }\AttributeTok{by =} \FunctionTok{c}\NormalTok{(}\StringTok{"iso3"} \OtherTok{=} \StringTok{"iso3c"}\NormalTok{))}
\end{Highlighting}
\end{Shaded}

Step 4: Inspect the result

\begin{Shaded}
\begin{Highlighting}[]
\FunctionTok{glimpse}\NormalTok{(world\_data\_joined)}
\end{Highlighting}
\end{Shaded}

\begin{verbatim}
## Rows: 257
## Columns: 9
## $ country_name    <chr> "Andorra", "United Arab Emirates", "Afghanistan", "Ant~
## $ iso3            <chr> "AND", "ARE", "AFG", "ATG", "AIA", "ALB", "ARM", "AGO"~
## $ country         <chr> "Andorra", "United Arab Emirates", "Afghanistan", "Ant~
## $ iso2c           <chr> "AD", "AE", "AF", "AG", NA, "AL", "AM", "AO", NA, "AR"~
## $ year            <dbl> 2020, 2020, 2020, 2020, NA, 2020, 2020, 2020, NA, 2020~
## $ gdp_per_capita  <dbl> 37361.0901, 37173.8754, 510.7871, 15360.4544, NA, 5370~
## $ life_expectancy <dbl> 79.41800, 81.93600, 61.45400, 77.16100, NA, 77.82400, ~
## $ population      <dbl> 77380, 9401038, 39068979, 91846, NA, 2837849, 2961500,~
## $ geometry        <MULTIPOLYGON [°]> MULTIPOLYGON (((1.53521 42...., MULTIPOLY~
\end{verbatim}

\begin{Shaded}
\begin{Highlighting}[]
\NormalTok{na\_count }\OtherTok{\textless{}{-}} \FunctionTok{sum}\NormalTok{(}\FunctionTok{is.na}\NormalTok{(world\_data\_joined}\SpecialCharTok{$}\NormalTok{population))}
\NormalTok{na\_count}
\end{Highlighting}
\end{Shaded}

\begin{verbatim}
## [1] 43
\end{verbatim}

\textbf{Saving Your Cleaned Data}

Step 1: Load required package

\begin{Shaded}
\begin{Highlighting}[]
\NormalTok{pacman}\SpecialCharTok{::}\FunctionTok{p\_load}\NormalTok{(sf)}
\end{Highlighting}
\end{Shaded}

Step 2: Define the output path

\begin{Shaded}
\begin{Highlighting}[]
\NormalTok{output\_filename }\OtherTok{\textless{}{-}} \StringTok{"world\_data\_cleaned\_joined.gpkg"}
\FunctionTok{message}\NormalTok{(}\StringTok{"Saving cleaned data to: "}\NormalTok{, output\_filename)}
\end{Highlighting}
\end{Shaded}

\begin{verbatim}
## Saving cleaned data to: world_data_cleaned_joined.gpkg
\end{verbatim}

Step 3: Write the GeoPackage

\begin{Shaded}
\begin{Highlighting}[]
\NormalTok{sf}\SpecialCharTok{::}\FunctionTok{st\_write}\NormalTok{(}
\NormalTok{  world\_data\_joined, output\_filename, }\AttributeTok{delete\_layer =} \ConstantTok{TRUE}
\NormalTok{)}
\end{Highlighting}
\end{Shaded}

\begin{verbatim}
## Warning in CPL_get_gdal_drivers(0): GDAL Error 1: libnetcdf.so.22: cannot open
## shared object file: No such file or directory
## Warning in CPL_get_gdal_drivers(0): GDAL Error 1: libnetcdf.so.22: cannot open
## shared object file: No such file or directory
\end{verbatim}

\begin{verbatim}
## Warning in CPL_get_gdal_drivers(0): GDAL Error 1: libmariadb.so.3: cannot open
## shared object file: No such file or directory
## Warning in CPL_get_gdal_drivers(0): GDAL Error 1: libmariadb.so.3: cannot open
## shared object file: No such file or directory
\end{verbatim}

\begin{verbatim}
## Warning in CPL_get_gdal_drivers(0): GDAL Error 1: libnetcdf.so.22: cannot open
## shared object file: No such file or directory
## Warning in CPL_get_gdal_drivers(0): GDAL Error 1: libnetcdf.so.22: cannot open
## shared object file: No such file or directory
\end{verbatim}

\begin{verbatim}
## Warning in CPL_get_gdal_drivers(0): GDAL Error 1: libmariadb.so.3: cannot open
## shared object file: No such file or directory
## Warning in CPL_get_gdal_drivers(0): GDAL Error 1: libmariadb.so.3: cannot open
## shared object file: No such file or directory
\end{verbatim}

\begin{verbatim}
## Deleting layer `world_data_cleaned_joined' using driver `GPKG'
## Writing layer `world_data_cleaned_joined' to data source 
##   `world_data_cleaned_joined.gpkg' using driver `GPKG'
## Writing 257 features with 8 fields and geometry type Multi Polygon.
\end{verbatim}

\begin{Shaded}
\begin{Highlighting}[]
\FunctionTok{message}\NormalTok{(}\StringTok{"Data saved successfully!"}\NormalTok{)}
\end{Highlighting}
\end{Shaded}

\begin{verbatim}
## Data saved successfully!
\end{verbatim}

\end{document}
